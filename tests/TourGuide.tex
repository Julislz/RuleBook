\section{Tour guide}
The robot guides spectators to the audience area and answer their questions after explaining what's @Home about.

\subsection{Focus}
This test focuses in safe outdoor navigation, people detection, gesture recognition, unconstrained natural language processing, and Human-Robot Interaction

\subsection{Setup}
\begin{itemize}
	\item \textbf{Location:} This test takes place outside the arena in a public space close to the @Home area.

	\item \textbf{Other people:} There are no restrictions on other people walking by or standing around throughout the complete task.

	\item \textbf{In Parallel:} This test can run in parallel, with several teams tested simultaneously.
\end{itemize}

\subsection{Task}

\begin{enumerate}
	\item \textbf{Start:} The robot waits at a designated starting position for the referee to give the start signal. When the referees start the time, the team is allowed to (briefly) provide some remarks about the robot's operation. After the instruction, the referee gives the start signal to the robot.\\

	\item \textbf{Finding spectators:} The robot starts moving to an open area and looks for (preferably large) groups of people. Once located the robot must approach to the spectators while calling for their attention in a \emph{friendly} way.\\

	People trying to call the attention of the robot (e.g.~by waving or shouting) have priority over those just walking by despite the number of the crowd. The robot may also approach to a single person.\\

	\item \textbf{Greeting an spectator:} Once the robot has gained the attention of the spectators, it must introduce itself (i.e.~saying it's name), and greet one of the spectators as customary in the venue's country (e.g.~bowing, handshaking, waving, etc).\\

	Note that all spectators may also want to greet the robot. The robot is expected to be polite and continue greeting on demand.\\

	\item \textbf{Guiding the spectators:} The robot must gently ask the spectators to follow it to any of the @Home audience areas and guide them there. Should the people not be willing to follow the robot, it must thank them and start looking for another group of spectators.\\

	\item \textbf{Explaining the league:} Once at the @Home audience area, the robot must ask the spectators to take seat. The robot proceeds to \textit{briefly} introduce RoboCup@Home and explain the League's objectives. \\

	\item \textbf{Answering questions:} At the end of the speech, the robot asks for questions from the spectators regarding what it just explained, answering at least two of them. The robot is allowed to rephrase questions before answering them.\\

\end{enumerate}

\subsection{Additional rules and remarks}
\begin{enumerate}
	\item \textbf{Safety First!} The robot will be stop at the slightest possibility of a human being harmed or molested. The robot must not force interaction with humans, nor scare them or make them feel uncomfortable. \\

	\item \textbf{Referee guard:} During the entire test, a referee will be following the robot from behind for keeping people safe and for scoring purposes.\\

	\item \textbf{Approaching to spectators:} When approaching to people the robot should act in a natural way by reducing its velocity as it approaches to the people. The robot must look safe and friendly.\\

	Shall the people flee, the robot must not chase them.\\

	\item \textbf{Spectators:} Spectators are people attending to the venue to see the competition with no restriction of any kind, therefore, their numbers, grouping, and behaviour are not controlled by the league. Were the case of no spectators available, volunteers can be used instead.\\

	\item \textbf{Bilingual robots:} Robots are allowed (and encouraged) to interact with people in a language other than English. In such cases, the robot must utter the English equivalent right after synthesising the localized sentence. \\

	Notice that spectators may prefer to ask questions in their native language when interacting with a bilingual robot. In such cases, the robot must translate the question for the Referee to understand it and answer the question in both languages.\\

	\item \textbf{Handshaking:} When handshaking, the robot must stay at a safe distance from the people (e.g.~about 1.5m) and reach out its \textit{hand}, but it must be a human, not the robot, who accepts and completes the handshake. If the human refuses to shake hands, the robot must retreat its manipulator immediately.\\

	\item \textbf{Disturbances from outside:} If a person from the audience (severely) interferes with the robot in a way that makes it impossible to solve the task, the team may repeat the test immediately.\\

	\item \textbf{Show must go on:} If the robot has engaged with a group of spectators when the allotted time for the test elapses, the robot is allowed to continue and finish the demonstration. However, no points are scored once the test is over.
\end{enumerate}

\subsection{Referee instructions}

The referees need to
\begin{itemize}
	\item Follow the robot at any time.
	\item Immediately stops the robot when considered necessary.
	\item Verify that the given answers are correct.
\end{itemize}

\subsection{OC instructions}

2h before test:
\begin{itemize}
	\item Recruit volunteers for the test (just in case).
	\item Announce the Start Location for the robots.
\end{itemize}

During the test:
\begin{itemize}
	\item Keep at least one area free in the audience area for robots to perform there.
	\item Send volunteers to join the Q\&A session to ask questions if necessary.
\end{itemize}

\newpage
\subsection{Score sheet}
The maximum time for this test is \textbf{10 minutes}.

\begin{scorelist}
	\scoreheading{Engaging spectators} % Max 50
	\scoreitem{30}{Find an spectator (or group)}
	\scoreitem{20}{Greet an spectator (handshake)}
	\scoreitem{10}{Greet and get greet by an spectator (bowing or waving)}

	\scoreheading{Guiding spectators} % Max 50
	\scoreitem{10}{Convince spectator to follow}
	\scoreitem{40}{Reach the audience area}

	\scoreheading{Q\&A Session} % Max 210
	% \scoreitem{10}{Finish talk without loosing spectators attention}
	\scoreitem{10}{Finish talk without loosing spectators}
	\scoreitem[2]{70}{Each correctly understood question}
	\scoreitem[2]{30}{Each correctly answered question}
	
	\scoreheading{Bilingual interaction} % Max 80
	\scoreitem{10}{Bilingual engaging}
	\scoreitem[2]{25}{Questions in $3^{rd}$ language}
	\scoreitem[2]{10}{Question answered also in $3^{rd}$ language}
	
	\setTotalScore{390}
\end{scorelist}


% Local Variables:
% TeX-master: "Rulebook"
% End:
