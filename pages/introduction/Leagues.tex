%% %%%%%%%%%%%%%%%%%%%%%%%%%%%%%%%%%%%%%%%%%%%%%%%%%%%%%%%%%%%%%%%%%%%%%%%%%%%
%%
%%    author(s): RoboCupAtHome Technical Committee(s)
%%  description: Introduction - Leagues
%%
%% %%%%%%%%%%%%%%%%%%%%%%%%%%%%%%%%%%%%%%%%%%%%%%%%%%%%%%%%%%%%%%%%%%%%%%%%%%%
\section{Leagues}
\label{sec:leagues}

\AtHome{} is divided into two Leagues. One of these grants complete freedom to all competitors with respect to the robot used, while in the other all competitors use the same robot. The official leagues and their names are:
\begin{itemize}
  \item \OPL
  \item \DSPL
\end{itemize}

\begin{wrapfigure}[21]{r}{0.30\textwidth}
	\vspace{-30pt}
	\begin{center}
		\includegraphics[width=0.25\textwidth]{images/toyota_hsr.png}
		\vspace{-10pt}
		\caption{Toyota HSR}
		\label{fig:toyota_hsr}
	\end{center}
\end{wrapfigure}
Each league focuses on a different aspect of service robotics by targeting specific abilities.

\subsection{Domestic Standard Platform League (DSPL)}

The main goal of the DSPL is to assist humans in a domestic environment, paying special attention to elderly people and people suffering from illness or disability.
As a consequence, the DSPL focuses on \AmI, \CV, \OM, safe indoor \NAV{} and \MAP, and \TP.
The robot used in the DSPL is the \HSR, shown in Figure \ref{fig:toyota_hsr}.

\subsection{Open Platform League (OPL)}

The OPL has had the same modus operandi since the foundation of \AtHome.
With no hardware constrains, OPL is the league for teams who want to test their own robot designs and configurations, as well as for old at-homers.
In this league, robots are tested to their limits without having in mind any concrete design restriction, although the scope is similar to the DSPL.
