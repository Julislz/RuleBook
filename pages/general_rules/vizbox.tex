\section{Audience interaction}
\label{rule:vizbox}

Part of making \RoboCup\AtHome{} appealing is to show the audience what robots should do and what they are actually doing during tasks.
In particular, providing information about what a robot is doing to the audience is important for the advancement of the league.
To this end, robots in \AtHome{} are expected to run the \RoboCup\AtHome{} \href{https://github.com/LoyVanBeek/vizbox}{VizBox}\footnote{\url{https://github.com/LoyVanBeek/vizbox}}, which is a web server to be run on a robot during a task.
The page it serves can be displayed on a screen and is visible to the audience via a secondary computer in or around the \Arena{}, which is connected to the web server via the wireless network.
The \iterm{VizBox} can:
\begin{itemize}
	\item display images of what a robot can see, such as camera images, or a visualization of the robot's world model, the robot's map, or anything else that clarifies what the robot is trying to do
	\item show an outline of the current tasks and the robot's current state in it
	\item display subtitles of the conversation between a robot and an operator
\end{itemize}
Additionally, the \iterm{VizBox} offers a way to input text commands to the robot so that automatic speech recognition can be bypassed, if necessary.

The documentation of the component is maintained in the \iterm{VizBox} repository.
All teams should ideally run the same VizBox code, as the audience should be shown a consistent presentation; however, opening a pull request to share any changes is much appreciated so that all teams can benefit from them.
