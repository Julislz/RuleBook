%%%%%%%%%%%%%%%%%%%%%%%%%%%%%%%%%%%%%%%%%%%%%%%%%%%%%%%%%
\section{Robots}
\label{rule:robots}

\subsection{Number of Robots}
\label{rule:robots_number}

\begin{enumerate}
	\item \textbf{Registration:} The maximum \term{number of robots} per team is \emph{two} (2).
	\item \textbf{Regular Tests:} Only one robot is allowed per test. For different test runs, different robots can be used.
	\item \textbf{\FINAL:} In the \FINAL, both robots can be used simultaneously.
\end{enumerate}

\subsection{Appearance and Safety}
\label{rule:robot_appearance}

Robots should have a product-like appearance and be safe to operate.
The following rules apply to all robots:
\begin{enumerate}
	\item \textbf{Cover:} The robot's internal hardware (electronics and cables) should be covered so that safety is ensured. The use of (visible) duct tape is strictly prohibited.
	\item \textbf{Loose cables:} Loose cables hanging out of the robot are not permitted.
	\item \textbf{Safety:} The robot must not have sharp edges or elements that might harm people.
	\item \textbf{Annoyance:} The robot must not be continuously making loud noises or use blinding lights.
	\item \textbf{Marks:} The robot may not exhibit any kind of artificial marks or patterns.
	\item \textbf{Driving:} To be safe, the robots should be careful when driving. Obstacle avoidance is mandatory.
\end{enumerate}
The compliance with these rules will be verified during \RobotInspection{} (see \ref{sec:robot_inspection}).

\subsection{Standard Platform Leagues}

Standardized platforms allow teams to compete under equal conditions by eliminating all hardware-dependent variables from the tasks; therefore, \emph{unauthorized} modifications and alterations to the robots are strictly forbidden.
This includes, but is not limited to, attaching, connecting, plugging, gluing, and taping components into and onto the robot, as well as, modifying or altering the robot structure.
Not complying with this rule leads to an immediate disqualification and penalization of the team (see~\refsec{rule:extraordinary_penalties}).
Robots are, however, allowed to \enquote{wear} clothes, have stickers (such as a sticker exhibiting the logo of a sponsor), or be painted (provided that the robot provider has approved that).

All modifications to the robots will be examined during the \RobotInspection{} (see \ref{sec:robot_inspection}).
Note that the EC and TC members may request re-inspection of an SPL robot at any time during the competition.

\subsubsection{Authorized DSPL Modifications}
\label{rule:osl_dspl}

In the \DSPL{}, teams may use an external laptop, which is safely located in the official \MountingBracket{} provided by Toyota and is connected to the \HSR{} via an Ethernet cable.
Any laptop fitting inside the \MountingBracket{} is allowed to be used, regardless of its technical specification.
Furthermore, teams are allowed to attach the following devices to either the \HSR{} or the laptop in the \MountingBracket:
\begin{itemize}
	\item \textbf{Audio}: A USB audio output device, such as a USB speaker or a sound card dongle.
	    \item \textbf{Microphone}: A USB or AUX external microphone. Wireless microphones are not allowed.
	\item \textbf{Wi-Fi adapter}: A USB-powered IEEE 802.11ac (or newer) compliant device.
	\item \textbf{Ethernet Switch}: A USB-powered IEEE 802.3ab (or newer) compliant device.
\end{itemize}
In all cases, a maximum of three such devices can be attached, such that they may not increase the robot's dimensions.
For this purpose, using short cables and attaching the devices to the laptop in the \MountingBracket{} is advised.


\subsection{Robot Specifications for the Open Platform League }
Robots competing in the RoboCup@Home Open Platform League must comply with security specifications in order to avoid causing any harm while operating.

\subsubsection{Size and Weight}
\label{rule:robots_size}

\begin{enumerate}
	\item \textbf{Dimensions:} The dimensions of a robot should not exceed the limits of an average door (\SI{200}{\centi\meter} by \SI{70}{\centi\meter} in most countries).
	The TC may allow the qualification and registration of larger robots, but, due to local restrictions, it cannot be guaranteed that the robots can actually enter the \Arena{}.
	In doubt, please contact the \LOC.
	\item \textbf{Weight:} There are no specific weight restrictions; however, the weight of the robot and the pressure it exerts on the floor should not exceed local regulations for the construction of offices and/or buildings which are used for living in the country where the competitions is being held.
	\item \textbf{Transportation:} Team members are responsible for quickly moving the robot out of the \Arena.
	If the robot cannot move by itself (for any reason), the team members must be able to transport the robot away quickly and easily.
\end{enumerate}

\subsubsection{Appearance}
\label{rule:robots_appearance}

OPL robots should have an appearance that resembles a safe and finished product rather than an early stage prototype.
This, in particular, means that the robot's internal hardware (electronics and cables) should be completely covered so that safety is ensured.
Please note that covering the robot's internal hardware with a t-shirt is not forbidden, but is not advised.

\subsubsection{Emergency Stop Button}
\label{rule:robots_emergency_button}

\begin{enumerate}
	\item \textbf{Accessibility and visibility:} Every robot has to provide an easily accessible and visible \EmergencyStop{} button.
	\item \textbf{Color:} The \EmergencyStop{} must be coloured red and be the only red button on the robot.
	The TC may ask the team to tape over or remove any other red buttons present on the robot.
	\item \textbf{Robot behavior:} When the \EmergencyStop{} button is pressed, the robot and all its parts must stop moving immediately.
\end{enumerate}

\subsubsection{Start Button}
\label{rule:start_button}

\begin{enumerate}
	\item \textbf{Requirements:} As explained in~\refsec{rule:start_signal}, teams that aren't able to carry out the default start signal (opening the door) have to provide a \StartButton{} that can be used to start tests.
	Teams need to announce this to the TC before every test that involves a start signal, including the \RobotInspection{}.
	\item \textbf{Definition:} The \StartButton{} can be any \enquote{one-button procedure} that can be easily executed by a referee (such as releasing the \EmergencyStop{}, a green button, or a software button in a graphical user interface).
\end{enumerate}

\noindent\textbf{Note:} All robot requirements will be tested during the \RobotInspection{} (see~\ref{sec:robot_inspection}).

% \subsubsection{Audio output plug}
% \label{rule:roobt_audio_out}

% \begin{enumerate}
% 	\item \textbf{Mandatory plug:} Either the robot or some external device connected to it \emph{must} have a \iterm{speaker output plug}. It is used to connect the robot to the sound system so that the audience and the referees can hear and follow the robot's speech output.
% 	\item \textbf{Inspection:} The output plug needs to be presented to the TC during the \iterm{Robot Inspection} test (see~\refsec{sec:robot_inspection}).
% 	\item \textbf{Audio during tests:} Audio (and speech) output of the robot during a test have to be understood at least by the referees and the operators.
% 	\begin{compactitem}
% 		\item It is the responsibility of the teams to plug in the transmitter before a test, to check the sound system, and to hand over the transmitter to next team.
% 		\item Do not rely on the sound system! For fail-safe operation and interacting with operators make sure that the sound system is not needed, e.g., by having additional speakers directly on the robot.
% \end{compactitem}
% \end{enumerate}




% Local Variables:
% TeX-master: "../Rulebook"
% End:
